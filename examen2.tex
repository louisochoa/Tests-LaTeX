%%%%%%%%%%%%%%%%%%%%%%%%%%%%%%%%%%%%%%%%%%%
%%%%       Plantilla tipo ArtIculo     %%%%
%%%%%%%%%%%%%%%%%%%%%%%%%%%%%%%%%%%%%%%%%%%
%%%%      Para exámenes de ASI         %%%%
%%%%%%%%%%%%%%%%%%%%%%%%%%%%%%%%%%%%%%%%%%%
%%%%      Creado por Luis G. Ochoa     %%%%
%%%%%%%%%%%%%%%%%%%%%%%%%%%%%%%%%%%%%%%%%%%

%%%%%%   TIPO DE DOCUMENTO: ArtIculo   %%%%%%
\documentclass[letterpaper,11pt]{article}

%%%%%%%  ESPECIFICACION DE PAQUETES   %%%%%
\usepackage[margin=2.5cm]{geometry}
\usepackage{changepage}
\usepackage[spanish,es-lcroman]{babel} \decimalpoint
\usepackage{graphicx} %Manejo de imagenes
\usepackage{enumerate}% http://ctan.org/pkg/enumerate

\usepackage{paralist} %
%\usepackage{enumitem}% http://ctan.org/pkg/enumitem
\renewcommand{\labelenumi}{\theenumi}
\usepackage{simbolos} %Math symbols
%%%%%%%%%%%%%%%%%%%%%%%%%%%%%%%%%%%%%%%%%%%%%%%%%%%%%
%%%%%%%    INICIO DEL CUERPO DEL DOCUMENTO    %%%%%%%
%%%%%%%%%%%%%%%%%%%%%%%%%%%%%%%%%%%%%%%%%%%%%%%%%%%%%
\begin{document}

\setlength{\parindent}{0pt}

Álgebra Superior I \;\;\;\;\;\;\;\;\;\;\;\;\;\;\;\;\;\;\;\;\;\;\;\;\;\;\;\;\;\;\;\;\;\;\;\;\;\;\;\;\;\;\;\;\;\;\;\;\;\;\;\;\;\;\;\;\;\;\;\;\;\;\;\;\;\;\;\;\;\;\;\;\;\;\;\;\;\;\;\;\;\;\;\;\;\;\;\;\;\;\;\;\;\;\;\;\;\;
Semestre 2021-1

  \rule{16.6cm}{0.3pt} % Linea horizontal
  \begin{center}
    \LARGE \textbf{\\Tarea-Examen 2: Relaciones y funciones.}
  \end{center}
  \rule{16.6cm}{0.3pt} \newline

  Grupo: 4010 \newline
  Luis Gonzalo Ochoa Rivera \newline \newline

  Resuelve los siguientes ejercicios justificando tu respuesta. \newline

%\begin{enumerate}
  %\item
  1. (5 puntos) Diga si las siguientes afirmaciones son verdaderas o falsas, justificando su respuesta. \newline

  \begin{adjustwidth}{1cm}{0cm}
    \begin{enumerate}[(a)]
      \item Si $R$ y $S$ son relaciones definidas sobre $A \neq \emptyset$ tales
        que $R \cup S$ es una relación de equivalencia sobre $A$, entonces tanto
        $R$ y $S$ son relaciones de equivalencia sobre $A$.

      \item Si $R$ y $S$ son relaciones definidas sobre $A \neq \emptyset$ tales
        que $R \cap S$ es una relación de equivalencia sobre $A$, entonces
        tanto $R$ y $S$ son relaciones de equivalencia sobre $A$. \newline

    \end{enumerate}
  %\end{adjustwidth}

    Ambas afirmaciones son falsas, veamos un contraejemplo. \newline

    Sean $A = \{ 1, 2 \}$, $R = \{ (1, 1), (1, 2), (2, 2) \}$ y $S = \{(1, 1) (2, 1), (2, 2) \}$; \\
    entonces $R \cup S = \{ (1, 1), (1, 2), (2, 1), (2, 2) \}$. \newline
    La cual, es una relación de equivalencia; ya que es reflexiva ($1 \sim 1$ y $2 \sim 2$),
    simétrica ($1 \sim 2$ y $2 \sim 1$) y transitiva ($1 \sim 2$, $2 \sim 1$ y $1 \sim 1$;
    $2 \sim 1$, $1 \sim 2$ y $2 \sim 2$).
    Pero $R$ no es simétrica porque $(2, 1) \notin R$ y análogamente $S$ no es simétrica. \newline
    Por lo tanto $R$ y $S$ no son relaciones de equivalencia. \newline

    Similarmente, notemos que $R \cap S = \{ (1, 1), (2, 2) \}$ es una relación de equivalencia.
    Y como ya vimos, $R$ y $S$ no son relaciones de equivalencia. \newline
  \end{adjustwidth}

  %\item
  2. (5 puntos) Sean $A, B, C, D$ conjuntos cualesquiera y sean
  $f : A \rightarrow B$, $g : B \rightarrow C \;$ y $\; h : C \rightarrow D$ funciones.
  Demuestre lo siguiente: \newline

  \begin{adjustwidth}{1cm}{0cm}
    \begin{enumerate}[(a)]
      \item  Si $f$ y $g$ son biyectivas, entonces $g \circ f$ es biyectiva. \newline
        Supongamos $f$ y $g$ son biyectivas. \newline
        Entonces $f^{-1}$ y $g^{-1}$ son funciones, por el Corolario 3.53
        y $f^{-1} \circ g^{-1}$ es una función por la Definición 3.40. \newline
        Entonces
        \begin{align*}
          (g \circ f) \circ (f^{-1} \circ g^{-1}) &= g \circ (f \circ f^{-1}) \circ g^{-1} \;\;
          por \;\; el \;\; Lema \;\; 3.48. \\
          &= g \circ id_{B} \circ g^{-1} \;\; por \;\; la \;\; Proposici\acute{o}n \;\; 3.52. \\
          &= g \circ g^{-1} \;\; por \;\; la \;\; Observaci\acute{o}n \;\; 3.43. \\
          &= id_{C} \;\; por \;\; la \;\; Proposici\acute{o}n \;\; 3.52.
        \end{align*}
        $i.e. \; (f^{-1} \circ g^{-1})$ es el inverso derecho de $g \circ f$ por la Definición 3.44. \newline

        Similarmente,
        \begin{align*}
          (f^{-1} \circ g^{-1}) \circ (g \circ f) &= f^{-1} \circ (g^{-1} \circ g) \circ f \\
          &= f^{-1} \circ id_{B} \circ f \\
          &= f^{-1} \circ f \\
          &= id_{A}
        \end{align*}
        $i.e. \; (f^{-1} \circ g^{-1})$ es el inverso izquierdo de $g \circ f$. \newline

        Entonces por la Definición 3.50, $g \circ f$ es invertible. \newline
        Y por el Corolario 3.53, $g \circ f$ es biyectiva. \newline

        Además, por la Proposición 3.52, $(g \circ f)^{-1}$ es función y es el inverso derecho e izquierdo de $\; g \circ f$. \newline
        Y por el Teorema 3.49, $(g \circ f)^{-1} = f^{-1} \circ g^{-1}$.

        \begin{flushright}
          $\square$
        \end{flushright}

      %\clearpage

      \item Si $g \circ f$ y $h \circ g$ son biyectivas, entonces $f$, $g$ y $h$
        son biyectivas. Sugerencia: demuestre primero que $g$ y $h$ son suprayectivas,
        luego que $f$ y $g$ son inyectivas, después que $h$ es inyectiva y finalmente
        que $f$ es sobre. \newline

        Supongamos $\; g \circ f \;$ y $\; h \circ g \;$ son biyectivas. \newline

        En particular, $\; g \circ f \;$ y $\; h \circ g \;$ son suprayectivas. \newline
        Sea $c \in C$, entonces $\exists a \in A$ tal que $c = g \circ f (a) = g(f(a))$. \newline
        Como $f$ es función, entonces $\exists b \in B \;$ t.q. $\; b = f(a) \;$ y $\; c = g(b)$. \newline
        $i.e. \; g$ es suprayectiva (ya que $\forall c \in C \;\; \exists b \in B \;$ t.q. $\; c = g(b)$). \newline

        Análogamente $h$ es suprayectiva. \newline

        $g \circ f \;$ y $\; h \circ g \;$ también son inyectivas. \newline

        Sean $b_1, b_2 \in B$ tales que $g(b_1) = g(b_2)$ \newline
        entonces $h(g(b_1)) = h(g(b_2))$, porque $h$ es función. \newline
        entonces $b_1 = b_2$, ya que $\; h \circ g \;$ es inyectiva. \newline
        Se sigue que $g$ es inyectiva. \newline

        Análogamente $f$ es inyectiva. \newline

        Sean $c_1, c_2 \in C$ tales que $h(c_1) = h(c_2)$. \newline
        Como $g$ es suprayectiva, $\exists b_1, b_2 \in B$ tales que $c_1 = g(b_1)$ y $c_2 = g(b_2)$. \newline
        Entonces $h(g(b_1)) = h(c_1) = h(c_2) = h(g(b_2))$, ya que $h$ es función. \newline
        Entonces $b_1 = b_2$, ya que $\; h \circ g \;$ es inyectiva. \newline
        Como $g$ es función, entonces $c_1 = g(b_1) = g(b_2) = c_2$ \newline
        $i.e. \; h$ es inyectiva. \newline

        Sea $b \in B$, como $g$ es función, entonces $\exists c \in C \;$ t.q. $\; c = g(b)$. \newline
        Como $\; g \circ f \;$ es suprayectiva, $\exists a \in A \;$ t.q. $\; g(b) = c = g(f(a))$. \newline
        Como $g$ es inyectiva, entonces $b = f(a)$. \newline
        Se sigue que $f$ es suprayectiva. \newline

        Como $f, g$ y $h$ son suprayectivas e inyectivas, entonces son biyectivas por la Definición 3.25. \newline

        \begin{flushright}
          $\square$
        \end{flushright}

    \end{enumerate}
  \end{adjustwidth}
%\end{enumerate}

\end{document}
%%%%%%%%%%%%%%%%%%%%%%%%%%%%%%%%%%%%%%%%%%%%%%%%%%
%%%%%%%    FIN DEL CUERPO DEL DOCUMENTO    %%%%%%%
%%%%%%%%%%%%%%%%%%%%%%%%%%%%%%%%%%%%%%%%%%%%%%%%%%
